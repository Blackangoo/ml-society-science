\subsection{Statistical testing\textsuperscript{*}}
\only<article>{A common type of decision problem is a statistical test. This arises when we have a set of possible candidate models $\CM$ and we need to be able to decide which model to select after we see the evidence.
  Many times, there is only one model under consideration, $\model_0$, the so-called \alert{null hypothesis}. Then, our only decision is whether or not to accept or reject this hypothesis.}
\begin{frame}
  \frametitle{Simple hypothesis testing}
  \only<article>{Let us start with the simple case of needing to compare two models.}
  \begin{block}{The simple hypothesis test as a decision problem}
    \begin{itemize}
    \item $\CM = \{\model_0, \model_1\}$
    \item $a_0$: Accept model $\model_0$
    \item $a_1$: Accept model $\model_1$
    \end{itemize}
    \begin{table}[H]
      \begin{tabular}{c|cc}
        $\util$& $\model_0$& $\model_1$\\\hline
        $a_0$ & 1 & 0\\
        $a_1$ & 0 & 1
      \end{tabular}
      \caption{Example utility function for simple hypothesis tests.}
    \end{table}
    \only<article>{There is no reason for us to be restricted to this utility function. As it is diagonal, it effectively treats both types of errors in the same way.}
  \end{block}

  \begin{example}[Continuation of the medium example]
    \begin{itemize}
    \item $\model_1$: that John is a medium.
    \item $\model_0$: that John is not a medium.
    \end{itemize}
    \only<article>{
      Let $x_t$ be $0$ if John makes an incorrect prediction at time $t$ and $x_t = 1$ if he makes a correct prediction. Let us once more assume a Bernoulli model, so that John's claim that he can predict our tosses perfectly means that for a sequence of tosses $\bx = x_1, \ldots, x_n$,
      \[
        P_{\model_1}(\bx) = \begin{cases}
          1, & x_t = 1 \forall t \in [n]\\
          0, & \exists t \in [n] : x_t = 0.
        \end{cases}
      \]
      That is, the probability of perfectly correct predictions is 1, and that of one or more incorrect prediction is 0. For the other model, we can assume that all draws are independently and identically distributed from a fair coin. Consequently, no matter what John's predictions are, we have that:
      \[
        P_{\model_0}(\bx = 1 \ldots 1) = 2^{-n}.
      \]
      So, for the given example, as stated, we have the following facts:
      \begin{itemize}
      \item If John makes one or more mistakes, then $\Pr(\bx \mid \model_1) = 0$ and $\Pr(\bx \mid \model_0) = 2^{-n}$. Thus, we should perhaps say that then John is not a medium
      \item If John makes no mistakes at all, then 
        \begin{align}
          \Pr(\bx = 1, \ldots, 1 \mid \model_1) &= 1,
          &
            \Pr(\bx = 1, \ldots, 1 \mid \model_0) &= 2^{-n}.
        \end{align}
      \end{itemize}
      Now we can calculate the posterior distribution, which is
      \[
        \bel(\model_1 \mid \bx = 1, \ldots, 1) = \frac{1 \times \bel(\model_1)}{1 \times \bel(\model_1) + 2^{-n} (1 - \bel(\model_1))}.
      \]
      Our expected utility for taking action $a_0$ is actually
    }
    \[
      \E_\bel(\util \mid a_0) = 1 \times \bel(\model_0 \mid \bx) + 0 \times \bel(\model_1 \mid \bx), \qquad
      \E_\bel(\util \mid a_1) = 0 \times \bel(\model_0 \mid \bx) + 1 \times \bel(\model_1 \mid \bx)
    \]
  \end{example}
  
\end{frame}


\begin{frame}
  \frametitle{Null hypothesis test}
  Many times, there is only one model under consideration, $\model_0$, the so-called \alert{null hypothesis}. \only<article>{ This happens when, for example, we have no simple way of defining an appropriate alternative. Consider the example of the medium: How should we expect a medium to predict? Then, our only decision is whether or not to accept or reject this hypothesis.}
  \begin{block}{The null hypothesis test as a decision problem}
    \begin{itemize}
    \item $a_0$: Accept model $\model_0$
    \item $a_1$: Reject model $\model_0$
    \end{itemize}
  \end{block}

  \begin{example}{Construction of the test for the medium}
    \index{policy!for statistical testing}
    \begin{itemize}
    \item<2-> $\model_0$ is simply the $\Bernoulli(1/2)$ model: responses are by chance.
    \item<3-> We need to design a policy $\pol(a \mid \bx)$ that accepts or rejects depending on the data.
    \item<4-> Since there is no alternative model, we can only construct this policy according to its properties when $\model_0$ is true.
    \item<5-> In particular, we can fix a policy that only chooses $a_1$ when $\model_0$ is true a proportion $\delta$ of the time.
    \item<6-> This can be done by construcing a threshold test from the inverse-CDF.
    \end{itemize}
  \end{example}
\end{frame}
\begin{frame}
  \frametitle{Using $p$-values to construct statistical tests}
  \begin{definition}[Null statistical test]
    \only<article>{
      A statistical test $\pol$ is a decision rule for accepting or rejecting a hypothesis on the basis of evidence. A $p$-value test rejects a hypothesis whenever the value of the statistic $f(x)$ is smaller than a threshold.}
    The statistic $f : \CX \to [0,1]$ is  designed to have the property:
    \[
      P_{\model_0}(\cset{x}{f(x) \leq \delta}) = \delta.
    \]
    If our decision rule is:
    \[
      \pol(a \mid x) =
      \begin{cases}
        a_0, & f(x) \geq \delta\\
        a_1, & f(x) < \delta,
      \end{cases}
    \]
    the probability of rejecting the null hypothesis when it is true is exactly $\delta$.
  \end{definition}
  \only<presentation>{The value of the statistic $f(x)$, otherwise known as the \alert{$p$-value}, is uninformative.}
  \only<article>{This is because, by definition, $f(x)$ has a uniform distribution under $\model_0$. Hence the value of $f(x)$ itself is uninformative: high and low values are equally likely. In theory we should simply choose $\delta$ before seeing the data and just accept or reject based on whether $f(x) \leq \delta$. However nobody does that in practice, meaning that $p$-values are used incorrectly. Better not to use them at all, if uncertain about their meaning.}
\end{frame}
\begin{frame}
  \begin{block}{Issues with $p$-values}
    \begin{itemize}
    \item They only measure quality of fit \alert{on the data}.
    \item Not robust to model misspecification. \only<article>{For example, zero-mean testing using the $\chi^2$-test has a normality assumption.}
    \item They ignore effect sizes. \only<article>{For example, a linear analysis may determine that there is a significant deviation from zero-mean, but with only a small effect size of 0.01. Thus, reporting only the $p$-value is misleading}
    \item They do not consider prior information. 
    \item They do not represent the probability of having made an error. \only<article>{In particular, a $p$-value of $\delta$ does not mean that the probability that the null hypothesis is false given the data $x$, is $\delta$, i.e. $\delta \neq \Pr(\neg \model_0 \mid x)$.}
    \item The null-rejection error probability is the same irrespective of the amount of data (by design).
    \end{itemize}
  \end{block}
\end{frame}

\begin{frame}
  \only<article>{Let us consider the example of the medium.}
  \begin{block}{$p$-values for the medium example}
    \begin{itemize}
    \item<2->$\model_0$ is simply the $\Bernoulli(1/2)$ model:
      responses are by chance. 
    \item<3->CDF: $P_{\model_0}(N \leq n \mid K = 100)$ \only<article> {is the probability of at most $N$ successes if we throw the coin 100 times. This is in fact the cumulative probability function of the binomial distribution. Recall that the binomial represents the distribution for the number of successes of independent experiments, each following a Bernoulli distribution.}
    \item<4->ICDF:  the number of successes that will happen with probability at least $\delta$
    \item<5->e.g. we'll get at most 50 successes a proportion $\delta = 1/2$ of the time.
    \item<6>Using the (inverse) CDF we can construct a policy $\pol$ that selects $a_1$ when $\model_0$ is true only a $\delta$ portion of the time, for any choice of $\delta$.
    \end{itemize}
  \end{block}
  \begin{columns}
    \setlength\fheight{0.33\columnwidth}
    \setlength\fwidth{0.33\columnwidth}
    \begin{column}{0.5\textwidth}
      \only<3,4,5,6>{\input{../figures/binomial-cdf.tikz}}      
    \end{column}
    \begin{column}{0.5\textwidth}
      \only<4,5,6>{\input{../figures/binomial-icdf.tikz}}
    \end{column}
  \end{columns}    
\end{frame}



\begin{frame}
  \frametitle{Building a test}
  \begin{block}{The test statistic}
    We want the test to reflect that we don't have a significant number of failures.
    \[
      f(x) = 1 - \textrm{binocdf}(\sum_{t=1}^n x_t, n, 0.5)
    \]
  \end{block}
  \begin{alertblock}{What $f(x)$ is and is not}
    \begin{itemize}
    \item It is a \textbf{statistic} which is $\leq \delta$ a $\delta$ portion of the time when $\model_0$ is true.
    \item It is \textbf{not} the probability of observing $x$ under $\model_0$.
    \item It is \textbf{not} the probability of $\model_0$ given $x$.
    \end{itemize}
  \end{alertblock}
\end{frame}
\begin{frame}
  \begin{exercise}
    \begin{itemize}
    \item<1-> Let us throw a coin 8 times, and try and predict the outcome.
    \item<2-> Select a $p$-value threshold so that $\delta = 0.05$. 
      For 8 throws, this corresponds to \uncover<3->{$ > 6$ successes or $\geq 87.5\%$ success rate}.
    \item<3-> Let's calculate the $p$-value for each one of you
    \item<4-> What is the rejection performance of the test?
    \end{itemize}
    \setlength\fheight{0.25\columnwidth}
    \setlength\fwidth{0.5\columnwidth}
    \only<2,3>{
      \begin{figure}[H]
        \centering
        \input{../figures/p-value-example-rejection-threshold.tikz}
        \caption{Here we see how the rejection threshold, in terms of the success rate, changes with the number of throws to achieve an error rate of $\delta = 0.05$.}
      \end{figure}
      \only<article>{As the amount of throws goes to infinity, the threshold converges to $0.5$. This means that a statistically significant difference from the null hypothesis can be obtained, even when the actual model from which the data is drawn is only slightly different from 0.5.}
    }
    \only<4>{
      \begin{figure}[H]
        \centering
        \input{../figures/p-value-example-rejection.tikz}
        \caption{Here we see the rejection rate of the null hypothesis ($\model_0$) for two cases. Firstly, for the case when $\model_0$ is true. Secondly, when the data is generated from $\Bernoulli(0.55)$.}
      \end{figure}
      \only<article>{As we see, this method keeps its promise: the null is only rejected 0.05 of the time when it's true. We can also examine how often the null is rejected when it is false... but what should we compare against? Here we are generating data from a $\Bernoulli(0.55)$ model, and we can see the rejection of the null increases with the amount of data. This is called the \alert{power} of the test with respect to the $\Bernoulli(0.55)$ distribution. }
    }
  \end{exercise}
\end{frame}

\begin{frame}
  \begin{alertblock}{Statistical power and false discovery.}
    Beyond not rejecting the null when it's true, we also want:
    \begin{itemize}
    \item High power: Rejecting the null when it is false.
    \item Low false discovery rate: Accepting the null when it is true.
    \end{itemize}
  \end{alertblock}
  \begin{block}{Power}
    The power depends on what hypothesis we use as an alternative.
    \only<article>{This implies that we cannot simply consider a plain null hypothesis test, but must formulate a specific alternative hypothesis. }
  \end{block}

  \begin{block}{False discovery rate}
    False discovery depends on how likely it is \alert{a priori} that the null is false.
    \only<article>{This implies that we need to consider a prior probability for the null hypothesis being true.}
  \end{block}

  \only<article>{Both of these problems suggest that a Bayesian approach might be more suitable. Firstly, it allows us to consider an infinite number of possible alternative models as the alternative hypothesis, through Bayesian model averaging. Secondly, it allows us to specify prior probabilities for each alternative. This is especially important when we consider some effects unlikely.}
\end{frame}

\begin{frame}
  \frametitle{The Bayesian version of the test}
  \begin{example}
    \begin{enumerate}
    \item Set $\util(a_i, \model_j) = \ind{i =
        j}$.
      \only<article>{This choice makes sense if we care equally about
        either type of error.}
    \item Set $\bel(\model_i) = 1/2$. \only<article>{Here we place an
        equal probability in both models.}
    \item $\model_0$: $\Bernoulli(1/2)$. \only<article>{This is the
        same as the null hypothesis test.}
    \item $\model_1$: $\Bernoulli(\theta)$,
      $\theta \sim \Uniform([0,1])$. \only<article>{This is an
        extension of the simple hypothesis test, with an alternative
        hypothesis that says ``the data comes from an arbitrary
        Bernoulli model''.}
    \item Calculate $\bel(\model \mid x)$.
    \item Choose $a_i$, where $i = \argmax_{j} \bel(\model_j \mid x)$.
    \end{enumerate}
    \label{ex:bayesian-compound-hypothesis-test}
  \end{example}
  \begin{block}{Bayesian model averaging for the alternative model $\model_1$}
    \only<article>{In this scenario, $\model_0$ is a simple point model, e.g. corresponding to a $\Bernoulli(1/2)$. However $\model_1$ is a marginal distribution integrated over many models, e.g. a $Beta$ distribution over Bernoulli parameters.}
    \begin{align}
      P_{\model_1}(x) &= \int_\Param B_{\param}(x) \dd \beta(\param) \\
      \bel(\model_0 \mid x) &= \frac{P_{\model_0}(x) \bel(\model_0)}
                              {P_{\model_0}(x) \bel(\model_0) + P_{\model_1}(x) \bel(\model_1)}
    \end{align}
  \end{block}
\end{frame}
\begin{frame}
  \only<1>{
    \begin{figure}[H]
      \centering
      \input{../figures/p-value-example-posterior.tikz}
      \caption{Here we see the convergence of the posterior probability.}
    \end{figure}
    \only<article>{As can be seen in the figure above, in both cases, the posterior converges to the correct value, so it can be used to indicate our confidence that the null is true.}
  }
  \only<2>{
    \begin{figure}[H]
      \centering
      \input{../figures/p-value-example-null-posterior.tikz}
      \caption{Comparison of the rejection probability for the null and the Bayesian test when $\model_0$ is true.}
    \end{figure}
    \only<article>{Now we can use this Bayesian test, with uniform prior, to see how well it performs. While the plain null hypothesis test has a fixed rejection rate of $0.05$, the Bayesian test's rejection rate converges to 0 as we collect more data.}
  }
  \only<3>{
    \begin{figure}[H]
      \centering
      \input{../figures/p-value-example-true-posterior.tikz}
      \caption{Comparison of the rejection probability for the null and the Bayesian test when $\model_1$ is true.}
    \end{figure}
    \only<article>{However, both methods are able to reject the null hypothesis more often when it is false, as long as we have more data.}
  }
\end{frame}

\begin{frame}
  \frametitle{Concentration inequalities and confidence intervals}
  \only<article>{
    There are a number of inequalities from probability theory that allow us to construct high-probability confidence intervals. The most well-known of those is Hoeffding's inequality \index{Hoeffding inequality|textbf}, the simplest version of which is the following:
    \begin{lemma}[Hoeffding's inequality]
      Let $x_1, \ldots, x_n$ be a series of random variables, $x_i \in [0, 1]$. Then it holds that, for the empirical mean:
      \[
        \mu_n \defn \frac{1}{n} sum_{t=1}^n x_t
      \]
      \begin{equation}
        \Pr(\mu_n \geq \E \mu_n+ \epsilon) \leq e^{-2n\epsilon^2}.
        \label{eq:hoeffding}
      \end{equation}
    \end{lemma}
    When $x_t$ are i.i.d, $\E \mu_n = \E x_t$. This allows us to construct an interval of size $\epsilon$ around the true mean. This can generalise to a two-sided interval:
    \[
      \Pr(|\mu_n - \E \mu_n| \geq \epsilon) \leq 2e^{-2n\epsilon^2}.
    \]
    We can also rewrite the equation to say that, with probability at least $1 - \delta$
    \[
      |\mu_n - \E \mu_n| \leq \sqrt{\frac{\ln 2/\delta}{2n}}
    \]
  }
\end{frame}
\begin{frame}
  \frametitle{Further reading}
  \begin{block}{Points of significance (Nature Methods)}
    \begin{itemize}
    \item Importance of being uncertain \url{https://www.nature.com/articles/nmeth.2613}
    \item Error bars \url{https://www.nature.com/articles/nmeth.2659}
    \item P values and the search for significance \url{https://www.nature.com/articles/nmeth.4120}
    \item Bayes' theorem \url{https://www.nature.com/articles/nmeth.3335}
    \item Sampling distributions and the bootstrap \url{https://www.nature.com/articles/nmeth.3414}
    \end{itemize}
  \end{block}
\end{frame}



%%% Local Variables:
%%% mode: latex
%%% TeX-master: "notes"
%%% End:
