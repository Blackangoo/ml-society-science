\chapter*{Preface}

This book is a technical introduction to important aspects of machine learning in science and society. It covers the most fundamental concepts in privacy, fairness, causality, reproducibility and experiment design. While I have tried to make the book as rigorous as possible, in the interest of brevity some technical details are omitted. 

However, even though the book is not addressed to a technical audience, every concept is necessarily connected to precise technical definition. This includes the bare minimum amount of theoretical development for a thorough understanding of the main concepts. Throughout the book, we use Bayesian decision theory and graphical models to offer a unifying perspective on those issues.


Throughout the book, we will use the following blocks of text for particular types of material.
\begin{exampleblock}{An example.}
Such blocks contain illustrations and examples. 
\end{exampleblock}

\begin{alertblock}{An important note.}
Important caveats or properties of algorithms will be given in such blocks.
\end{alertblock}

\begin{theoryblock}{Theoretical note.}
Discussions of a more theoretical nature, or notes that require some additional thinking will be placed in such blocks.
\end{theoryblock}

\begin{exerciseblock}{Exercise.}
  These exercises are meant to be done individually, either while reading the text, or in class.
\end{exerciseblock}

\begin{groupactivity}{Group activity.}
  These activities are meant to be done in class.
\end{groupactivity}

The first chapter offers an introduction to the field of machine learning in general and to fairness, privacy and reproducibility in particular. Chapter~\ref{ch:privacy} discusses privacy. Chapter~\ref{ch:fairness} talks about fairness.

I whole-heartedly recommend the book ``The Ethical Algorithm'' as a non-technical companion to this book and ``The foundations of differential privacy'' as a more thorough overview of differential privacy.

